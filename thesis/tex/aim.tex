\section{Aim and scope of the thesis}

\remark{The aim of this thesis is to develop a library for fast,
approximate \textit{maximum inner product search} (MIPS).}

\remark{To this end we provide a set of tools that would help machine
learning developers and practitioners create faster, more scalable machine learning models.}
Problems requiring MIPS naturally arise in many machine learning models, for example, in
classification problems with large output spaces, when linear models or deep neural networks are used.

Generally speaking, MIPS is the task involving searching through a massive collection of items for
an item most similar to a particular one, called query. The items may have many properties (features)
according to which items can be more related or differ from each other.
Every item is mathematically represented by a vector, with all vector's components being item's features.

Unfortunately, when one has to compare the query with hundreds of thousands of vectors,
a linear search time can become prohibitively expensive.
In other words comparing query and every database vector may not be feasible in practical applications.
To counter that, proposals were made in several articles to build a kind of an \textit{index} for speeding
up the searching time.
This index is essentially a specific data structure built by preprocessing database vectors (before
any of the queries are answered --- this is a one-time cost).
The most important feature of the index is short prediction (searching) time, training (index preparation)
can last much longer if needed.
It is not always necessary for the exact result to be returned --- it can often be approximate.
In this thesis we focus on such algorithms --- they trade some prediction precision for speed.

\remark{The library implemented in this thesis delivers three different MIPS algorithms,
recently introduced in the literature. Następnie w jednym lub dwóch zdaniach opisujecie każdy z tych trzech algorytmów}.

\remark{Moreover, the library not only offers high performance, but is also easy-to-use.
It provides a set of wrappers around a range of machine learning libraries, that allow users to
speed up their models without ever touching any of the implemented MIPS algorithms.} Consequently,
these wrappers can be used as drop-in replacements for a number of models in Python machine learning ecosystem.
We show some examples of how this library can be used in practice,
as well as detailed experiments measuring the speed and accuracy of our algorithms.

In particular we provide plugins for \texttt{Pytorch} (a deep learning library for Python), and \texttt{scikit-learn}
(a general-purpose machine learning library, also for Python), as well as a modified version of \texttt{FastText}
(a C++ library for efficient text classification).

All experimental results presented in this thesis were obtained using Amazon Elastic Compute Cloud (EC2).
EC2 is a part of Amazon.com's cloud-computing platform, Amazon Web Services (AWS). It allows users to rent virtual
computers on which they can run their own computer applications.

We have compared the implemented algorithms against the FAISS library \cite{faiss}, which is a library for efficient
similarity search and clustering of dense vectors, developed by Facebook AI Research. We refer the reader to
\cite{JDH17} for further details regarding this software.

The rest of this thesis is structured as follows.
\remark{The next chapter is dedicated to the theoretical description of the MIPS algorithms.
The third chapter describes implementation details of the library.
Chapter 4 presents experimental results.
The last chapter concludes the thesis.
In the appendix we describe how to run the library.}

\remark{The list below shows in detail the contribution of each author of this thesis:
\begin{itemize}
    \item \textbf{Marcin Elantkowski} wrote wrappers exposing C++ indexes to Python, prepared datasets used in the
        experiments, adapted the code so it could be used from Pytorch and FastText,
            and ran final tests on Amazon Web Services
    \item \textbf{Adam Krasuski} refactored and optimized the C++ code of the core algorithms,
        implemented an interface to the Faiss library, analyzed and described the empirical results.
    \item \textbf{Agnieszka Lipska} implemented the first version of the quantization algorithm,
        implemented the Faiss interface, and adapted the code to use it from Scikit-learn.
    \item \textbf{Franciszek Walkowiak} implemented the first versions of ALSH and
        hierarchical \mbox{k-means} and conducted preliminary tests on a personal computer.
\end{itemize}
}

