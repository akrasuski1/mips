Since we wanted our library to be acceessible to a wider public, we have prepared
a set of tools to help users apply our algorithms in real world settings. More specifically,
we have focused on three main aspects: creating bindings to a high-level programming language,
plugging our algorithms to popular tools like \texttt{scikit-learn}, and providing a set
of examples of how our code can be used.

We have decided to prepare bindings to a Python language. To do this, we used
the previously mentioned \texttt{pybind11} library. These bindings expose the core methods that
each of our algorithms expose, namely \texttt{train(...), add(...)} and \texttt{search(...)}.
We note however that it is trivial to expand those bindings to also expose internal
structs and methods of each class.

When it comes to using these bindings in real-world settings, we have prepared examples
in two major machine learning frameworks available in Python --- \texttt{pytorch}
and \texttt{scikit-learn}.

In \texttt{scikit-learn} all linear classifiers, regardless of the way they are trained
or loss they minimize, inherit from a single base class --- \texttt{LinearClassifierMixin}.
It was therefore enough to provide our own implementation of this mixin. This implementation
only differs in behaviour at prediction time --- it first trains an internal index on the
weight matrix, and then uses it to efficiently return top-$k$ highest rank classes.

In \texttt{PyTorch} we provide an implementation of a \texttt{ApproximateLinear} layer.
At train time this layer falls back to standard \texttt{pytorch} implementation, meaning
it can utilize the GPU acceleration. However, when prediction mode is enabled, we first
automatically train the index on the layer's weight matrix, and then provide a user with
a choice --- one can either use the layer as a standard \texttt{Linear}, or use it
to retrieve approximate top-$k$ output values.

We also provide a patch that adds our index to \texttt{FastText} text classification model.
We have used this model to perform experiments on Wiki-LSHTC and Amazon-3M datasets.
Since \texttt{FastText} is a command-line only utility, we have added \texttt{train-index}
and \texttt{approximate}-\texttt{predict} options available to users.
