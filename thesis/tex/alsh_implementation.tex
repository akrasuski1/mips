Hash tables are stored in a \texttt{std::vector} of structures, each containing
randomly generated hash constants and a \texttt{std::map} mapping
metahash key (implemented by \texttt{uint64\_t}) to \texttt{std::set} of vector indices.
In order to calculate metahash, $K$ single hashes as described in chapter 2.3 are computed.
Then, all these hashes are combined into one metahash key (\texttt{seed}) using an approach from the
\texttt{boost} library. Adding every hash value \texttt{v} to existing \texttt{seed} is done
by instruction containing a magic number:
\begin{verbatim}
seed ^= v + 0x9e3779b9 + (seed<<6) + (seed>>2);
\end{verbatim}

When answering a query, colliding vectors are found in all hash
tables and placed in an \texttt{std::unordered\_set}. Next, an exact inner product (vector's score)
between query and the colliders is calculated by the \path{faiss::fvec_inner_product} procedure from Faiss library.
Score processing is identical as in quantization, except best vectors are not returned but
written to resultant \texttt{FlatMatrix}. Truncation is necessary only in case there were more
colliders than $k$.
If less than $k$ vectors collided, remaining fields are filled with $-1$'s as required by
Faiss API.
