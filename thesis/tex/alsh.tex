The Asymmetric Locality Sensitive Hashing (ALSH) \cite{alsh} algorithm depends on hash functions
in order to determine similarities between vectors themselves and between vectors and queries.
Hash function transforms its possibly long input such as file (or vector of numbers like in our case)
into much smaller domain of output values.
A collision happens when two different inputs cause the function to return the same output.
In this algorithm's case it is a desired property for vectors to collide if they are in some way similar to each other.
\par
Single function may not be able to distingiush similarities among all vectors correctly i.e. some
significantly differing vectors might collide. Because of that $K$ functions grouped into
meta-hash function are used in this approach. Each function detects similarities of various aspects of vectors.
As these functions are partially random, aspects they will represent are not deterministic.
After a vector is hashed by all $K$ functions, outputs taken together form a hash table key.
An entry in hash table consists of key pointing to a set of vector indices for which the meta-hash function
returned this exact key. In other words, vectors are placed in ,,buckets'' --
every bucket represents one value of meta-hash.
To allow even higher abstraction level, $L$ meta-hash functions can be used to create $L$ hash tables --
vectors will be placed in $L$ sets of buckets according to some rules depending on applied meta-hash function.
Before hashing, vectors can be augmented using one of approaches presented in the beginning
of this section, e.g. Shrivastrava's or Neyshabur's transformation.
\par
Searching the index with query can as well be preceded by its earlier transformation.
Detecting which database vectors are most similar to query, relies on checking with which
vectors the query collided after being hashed using exactly the same meta-hash functions.
Speaking the bucket metaphor: in each bucket set we try to fit the query to some bucket and when
we succeed, this bucket contains vectors we are interested in.
Vectors' indices acquired this way are collected from every hash table and form candidate set from which top-$k$
most similar vectors should be chosen. In order to do that, exhaustive search can be performed.
Another idea is to create a ranking sorted by number of tables in which a collision
between query and vector occurred, although this seems computationally expensive.
\par
In the $i^{th}$ hash table the meta-hash function contains $K$ functions: $h_{1,i},$ $h_{2,i},$ $...,$ $h_{K,i}$.
By hashing the vector $x_j$ we will get the key $H_{i,j} = $ $[h_{1,i}(x_j),$ $h_{2,i}(x_j),$
$..., h_{K,i}(x_j)]$ under which the index of $j^{th}$ vector of database should be
placed in $i^{th}$ hash table. The following equation describes implemented hash function.
$N$ denotes Gaussian distribution and $U$ denotes random uniform distribution.
\begin{equation*}
h_{i,j}(\bm{x}) = \floor*{\frac{\bm{a_{i,j}^T x} + b_{i,j}}{r}}, \bm{a_{i,j}} \sim N(0,1), b_{i,j} \sim U(0,r)
\end{equation*}
\par
Random vectors $a_{i,j}$ and scalars $b_{i,j}$ are generated once and used throughout whole
algorithm, also in the querying phase. Parameters specific to this algorithm are $L$, $K$ and $r$.
Preparation of $L = 2$ hash tables for a database having 4 vectors:
$x_1, x_2, x_3$ and $x_4$ is presented below. Computations of vectors' keys in first table:\\
$H_{1,1} = [h_{1,1}(x_1), h_{2,1}(x_1), h_{3,1}(x_1)] = [0, 1, 0]$\\
$H_{1,2} = [h_{1,1}(x_2), h_{2,1}(x_2), h_{3,1}(x_2)] = [-1, 1, 0]$\\
$H_{1,3} = [h_{1,1}(x_3), h_{2,1}(x_3), h_{3,1}(x_3)] = [1, 0, -1]$\\
$H_{1,4} = [h_{1,1}(x_4), h_{2,1}(x_4), h_{3,1}(x_4)] = [0, 1, 0]$\\
Computations of vectors' keys in first table:\\
$H_{2,1} = [h_{1,2}(x_1), h_{2,2}(x_1), h_{3,2}(x_1)] = [2, 1, 0]$\\
$H_{2,2} = [h_{1,2}(x_2), h_{2,2}(x_2), h_{3,2}(x_2)] = [2, 1, 0]$\\
$H_{2,3} = [h_{1,2}(x_3), h_{2,2}(x_3), h_{3,2}(x_3)] = [2, 1, 0]$\\
$H_{2,4} = [h_{1,2}(x_4), h_{2,2}(x_4), h_{3,2}(x_4)] = [1, 0, -1]$\\
Hash tables will then look like this:
\renewcommand{\arraystretch}{1.2}
\begin{center}
\begin{tabular}{|l|c|}
\hline
\multicolumn{2}{|c|}{$L = 1$} \\
\hline
$[0, 1, 0]$ & $\{1, 4\}$ \\
\hline
$[-1, 1, 0]$ & $\{2\}$ \\
\hline
$[1, 0, -1]$ & $\{3\}$ \\
\hline
\end{tabular}
$\ \ \ \ $
\begin{tabular}{|l|c|}
\hline
\multicolumn{2}{|c|}{$L = 2$} \\
\hline
$[2, 1, 0]$ & $\{1, 2, 3\}$ \\
\hline
$[1, 0, -1]$ & $\{4\}$ \\
\hline
\end{tabular}
\end{center}

An example process of hashing the query is given below.\\
$H_{1,q} = [h_{1,1}(q), h_{2,1}(q), h_{3,1}(q)] = [0, 1, 0]$\\
$H_{2,q} = [h_{1,2}(q), h_{2,2}(q), h_{3,2}(q)] = [1, 0, -1]$

$H_{1,q}$ i $H_{2,q}$ are the keys with which one should look for vectors in first and second hash table respectively.
In the example, $x_1$ and $x_4$ turned out to be similar to $q$ according to first table, but only $x_4$ appears to be similar to query according to second table.
If the collision ranking idea is used, $x_4$ is the most similar vector to query, because the collision happened in both tables.
