\section{Introduction to data science and machine learning}

Today’s world is data-rich. 
People use large and rapidly growing volumes of data in many areas. 
\remark{It is called a big data: nie do końca rozumiem} and it can appear in different shapes and sizes and has many uses such as: targeting customers, improving healthcare and optimising business processes. 
It is almost impossible to analyze so big portions of information using traditional data processing software.
Machine learning comes as solution, giving an alternative to analyzing huge volumes of data.
This field of computer science became more and more popular because of large datasets, which are used almost everywhere.

Machine learning is a data science technique, which enables computers to learn like humans or animals.
It means that machines use algorithms to learn from data how to find a particular information and
they are not explictly programmed to do that.
That type of algorithms improves its operation when the number of training samples increases.
Machine learning has changed a lot in recent years because of new computer technologies.
Many algorithms have been around for a long time but the ability to apply mathematical calculations to big data faster is still a new development.

With the rise in big data it is used in such areas: natural language processing, image processing, recommendation systems and computational biology. Machine learning is applied in instances that we use in ordinary life like online recommendation systems --- friends suggestions on Facebook, Netflix movies suggestions based on our earlier choices.
It is also used in automatic article tagging on Wikipedia thanks to learned
associations between articles and categories assigned to them.

Machine learning uses two popular methods: supervised learning and unsupervised learning.
Supervised learning \remark{\sout{train model}} uses known input and output data so that it can predict outputs for future inputs.
It uses classification and regression methods to develop models.
Classification techniques classify input data into categories, they are used in medical imaging and speech recognition.
Regression techniques predict continuous responses, example usage can be predicting house prices. Unsupervised learning groups and interprets data based only on given input.
The most common unsupervised learning technique is clustering,
which analyses input data to find patterns in it. Typical usage of clustering method are recommendation systems.

The most visible difference between human learning and machine learning is speed. 
Thomas H. Davenport, an analyst, said that ``humans can typically create one or two good models a week; machine learning can create thousands of models a week'' \cite{machine_learning}.
This sentence proves that it is essential for scientists to develop their skills in that field because it increases the speed of developing new models.
In our thesis we try to implement a solution to problem called MIPS using supervised learning classification method.
