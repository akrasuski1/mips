\section{Introduction to data science and machine learning}

Machine learning is a data science technique, which enables computers to learn.
It means that machines use algorithms to learn from data how to find a particular information and
they are not explicitly programmed to do that.
That type of algorithms improves its operation when the number of training samples increases so they are used to analyze large volumes of data.

With the rise of big data machine learning algorithms are used in such areas: natural language processing, image processing, recommendation systems and computational biology. Machine learning is applied in instances that we use in ordinary life like online recommendation systems --- friends suggestions on Facebook, Netflix movies suggestions based on our earlier choices.
It is also used in automatic article tagging on Wikipedia thanks to learned
associations between articles and categories assigned to them.

Machine learning uses two popular methods: supervised learning and unsupervised learning.
Supervised learning uses known input and output data so that it can predict outputs for future inputs.
It uses classification and regression methods to develop models.
Classification techniques classify input data into categories, they are used in medical imaging and speech recognition.
Regression techniques predict continuous responses, example usage can be predicting housing prices. Unsupervised learning groups and interprets data based only on given input.
The most common unsupervised learning technique is clustering,
which analyses input data to find patterns in it. 
Typical usage of clustering method are recommendation systems.

In our thesis we consider problems with a large number of labels and we implement supervised learning classification algorithms.
