[Marcin: Write about Amazon etc. here]

% also write about what "precision" exactly means (intersection etc.)

% also write what IVF is (Inverted File, from Faiss, hierarchical K-Means with 1 layer)
% and state that it is for benchmark comparison

\subsection{Analysis of results}

The amount of data collected is quite large, so directly plotting the results would be
unreadable. Instead, we processed the data selecting, for each algorithm and dataset,
the nondominated parameter choices --- that is those, for which there is no other
parameter choice that gives results that are better in both query time and precision.

Algorithms were always called with request for top 100 vectors, but we recorded
precision at a couple of points: 1, 5, 25 and 100. It turns out the precision vs time
plots are very similar for each of those values - so to save space, we only show
graphs for precision at 100.

\begin{figure}[H]
	\centering
	\subfloat{
		\includegraphics[width=.48\textwidth]{../graphs/nondominated/Amazon-3M-p100.eps}
	}
	\subfloat{
		\includegraphics[width=.48\textwidth]{../graphs/nondominated/sift-p100.eps}
	}
	\\
	\subfloat{
		\includegraphics[width=.48\textwidth]{../graphs/nondominated/WikiLSHTC-p100.eps}
	}
	\subfloat{
		\includegraphics[width=.48\textwidth]{../graphs/nondominated/siftsmall-p100.eps}
	}
\caption{Pareto frontiers for each algorithm.}
\end{figure}

The graphs show some similarity - in all cases K-Means and IVF algorithms were much better
than the alternatives. ALSH algorithm gave very poor results on Amazon-3M and Wiki-LSHTC
datasets --- apparently sift vectors have some regularity that ALSH could exploit better.
Still, the precision did not exceed approximately $0.60$ for reasonable speedups even on
this dataset. Finally, the quantization-based algorithm, although it was able to give
reasonable precision, it was so slow, that the only dataset it did not time 
out\footnote{Timeout was set to twice the time of brute force search.}
was siftsmall dataset. It contained only ten thousand vectors though and was included
just for completion.

IVF and K-Means curves follow each other quite closely --- it could be expected though,
since K-Means algorithm is generalization of IVF. In almost all places K-Means are even
better than IVF --- again, this can be explained by having more parameters to be tested.
To check whether this is caused by possibly faster implementation or the introduction of
hierarchy, we split K-Means curve into three lines, each corresponding to a single
number of layers (for example, KMeans-2 has two layers).

\begin{figure}[H]
	\centering
	\subfloat{
		\includegraphics[width=.48\textwidth]{../graphs/nondominated-km/Amazon-3M-p100.eps}
	}
	\subfloat{
		\includegraphics[width=.48\textwidth]{../graphs/nondominated-km/sift-p100.eps}
	}
	\\
	\subfloat{
		\includegraphics[width=.48\textwidth]{../graphs/nondominated-km/WikiLSHTC-p100.eps}
	}
	\caption{Pareto frontiers for K-Means (split) and IVF.}
\end{figure}

From these graphs, we can see that in general IVF is approximately $30\% - 50\%$ faster
(on horizontal axis) than K-Means when using just a single layer. Both algorithms are 
equivalent in this case, so this means Faiss library is more optimized for this parameter 
choice\footnote{
Checking its source we can see it performs inner products of each query with all centroids
(from the first and only layer) all at once, before continuing to answer each query 
using its candidates. This allows one to use heavily optimized matrix-matrix multiplication
routines for the first phase, with additional memory locality advantages. 
In principle, this optimization
could be implemented for hierarchical K-Means as well (for the first layer only).
}.
In contrast, K-Means are much better when allowed to use two or three layers. Addition
of hierarchy speeds calculation to the level of IVF --- for some parameters even surpassing
it, especially for Amazon-3M dataset.

The difference between two-layer K-Means and three-layer ones is not as amplified. Since
training time (not pictured here) usually rises quite significantly with number of layers,
the minor additional speedup may not be worth it in practice.

It is also notable that two- and three-layer K-Means curves continue much further left
than IVF. Of course, such quick results come with significant impact on accuracy, but
IVF cannot even reach this speed. In a sense, its accuracy is zero there.
A possible explanation for this cutoff comes from its complexity analysis as described
in introduction to K-Means algorithm chapter. No matter what precision is, the minimal
complexity of flat K-Means (equivalent to IVF) is $O(\sqrt{N})$. Two- or three-layer
K-Means bring this limit down to $O(\sqrt[3]{N})$, or $O(\sqrt[4]{N})$ 
respectively\footnote{Admittedly with a small multiplicative slowdown.}.

TODO: table with nondominated parameters, additional analysis



